\documentclass{beamer} % "Beamer" is a word used in Germany to mean video projector. 

\usetheme{Montpellier} % Search online for beamer themes to find your favorite or use the Berkeley theme as in this file.

\usecolortheme{rose}
\usepackage[utf8x]{inputenc}
\usepackage[english,russian]{babel}
\usepackage{color} % It may be necessary to set PCTeX or whatever program you are using to output a .pdf instead of a .dvi file in order to see color on your screen.
\usepackage{graphicx} % This package is needed if you wish to include external image files.
\usepackage{verbatim}
\usepackage{amsmath}
\usepackage{ulem}
\usepackage{enumitem}
\setlist[itemize]{label=\textbullet}
\theoremstyle{definition} % See Lesson Three of the LaTeX Manual for more on this kind of "proclamation."
\newtheorem*{dfn}{A Reasonable Definition}               

\newcommand\smallerFont{\fontsize{9pt}{10.8}\selectfont}
\beamertemplatenavigationsymbolsempty
\setbeamerfont{page number in head/foot}{size=\large}
\setbeamertemplate{footline}[frame number]

\def\putImg<#1>#2{ \includegraphics<#1>[width=\textwidth]{../img/#2} }
%\newcommand<>{\putImg}[1] { \includegraphics#2[width=\textwidth]{pics/#1} }
\newcommand{\twopartdef}[4]
{
	\left\{
		\begin{array}{ll}
			#1 & \mbox{если } #2 \\
			#3 & \mbox{если } #4
		\end{array}
	\right.
}
\title{Оптимизация параметров стратегий поиска объектов на море}
\author{Антон Ковшаров\\{\small Научный руководитель: Ковалев А.С.}} 
\institute{Санкт-Петербургский национальный исследовательский университет \\ информационных технологий, механики и оптики}
\date{16 июня 2015} 

\AtBeginSection[]  % The commands within the following {} will be executed at the start of each section.
{
\begin{frame} % Within each "frame" there will be one or more "slides."  
\frametitle{Содержание} % This is the title of the outline.
\tableofcontents[currentsection]  % This will display the table of contents and highlight the current section.
\end{frame}
} %  Do not include the preceding set of commands if you prefer not to have a recurring outline displayed during your presentation.

\begin{document}

\beamertemplatetransparentcoveredmedium
\begin{frame} 
\titlepage
\end{frame}

\section{Постановка задачи}
\begin{frame}
  \frametitle{Цель работы}
   Построить маршрут поиска объекта максимизирующий вероятность его обнаружения. Фиксированы: \\
\begin{itemize}
  \item распределениe вероятности (зависимость от времени)
  \item параметры средства поиска
  \item стратегия поиска --- ``параллельное галсирование''
\end{itemize}
\end{frame}

\begin{frame}
  \frametitle{Стратегия поиска}
\putImg<+>{pic05-parallel_tacks.png}
\end{frame}

\begin{frame}
  \frametitle{Распределение вероятности}
\begin{columns}
\column{.6\textwidth}
\putImg<+>{pic03-1.png}
\putImg<+>{pic03-2.png}
\putImg<+>{pic03-3.png}
\putImg<+->{pic03-4.png}
\column{.4\textwidth}
\setbeamercovered{dynamic}
\begin{itemize}
\item<.(-3)>{Начальное распределение}
  \begin{itemize}
     \item Нормальное распределение
     \item Равномерное распределиние
  \end{itemize}
\item<.(-2)-.(0)>{Эволюция распределения (диффузия)}
\end{itemize}
\end{columns}
\end{frame}

\begin{comment}
\begin{frame}[t]
\only<1>{Распределение вероятности}
\only<2->{\sout{Распределение вероятности}\\}
\only<2->
{
Распределение частиц\\
\pause
\begin{itemize}
  \pause
\item частица --- гипотеза изначального положения объекта поиска и его поведения
  \pause
\item вес частицы --- вероятность гипотезы
  \pause
\item перемещение частиц с течением времени
  \pause
\item сбор частиц средством поиска
  \pause
\item больше вес собранных частиц --- больше вероятность обнаружить объект
\end{itemize}
}
\end{frame}
\end{comment}

\begin{comment}
\begin{frame}
  \frametitle{Параметры распределения}
\begin{itemize}
\item $A_{t_0} : R^2 \to R $ --- начальное распределения (аппроксимируется матрицей)
\item $f(A_t, \Delta t) = A_{t+\Delta t}$ --- функция изменения распределения
\end{itemize}
\end{frame}
\end{comment}

\begin{comment}
\begin{frame}
  \frametitle{Параметры средства поиска}
\begin{itemize}
\item $p_0$ --- начальное положение средства поиска
\item $v$ --- cкорость средства поиска
\item $r$ --- радиус обнаружения средства поиска (все частицы попавшие в круг радиуса обнаружения считаются ``собранными'')
\end{itemize}
\end{frame}
\end{comment}

\begin{comment}
\begin{frame}
  \frametitle{Параметры стратегии поиска}
  Параллельное галсирование
\begin{itemize}
  \item $l$ --- прямая параллельная направлению галсов
\end{itemize}
\end{frame}
\end{comment}

\begin{frame}
\begin{columns}
\column{.6\textwidth}
\putImg<+>{pic04-1.png}
\putImg<+>{pic04-2.png}
\putImg<+>{pic04-3.png}
\column{.4\textwidth}
\begin{itemize}
    \item<.(-2)-.(-1)>{Построение маршрута поиска объекта, основываясь на поле вероятности}
    \item<.(0)>{Симуляция прохождения маршрута}
\end{itemize}
\end{columns}
\end{frame}

\begin{frame}
  \frametitle{Входные данные}
\begin{itemize}
\item параметры распределения
\item параметры средства поиска
\item параметры стратегии поиска
\item время поиска
\end{itemize}
\end{frame}

\begin{frame}
  \frametitle{Задача}
\begin{itemize}
  \item {$\pi : \Pi$ --- частица}
  \item {$w_{\pi}$ --- вес частицы (сумма весов 1)}
  %\item {$pos(w, t)$ --- позиция частицы $w$ в момент времени $t$}
  %\item {$posFinder(t)$ --- позиция средства поиска}
  \item {$\chi(\pi) = 
	\left\{
		\begin{array}{ll}
			1 & \mbox{если } {\exists t\ {dist(posFinder(t), pos(\pi, t)) <= r}}\\
			0 & \mbox{иначе }
		\end{array}
	\right.$
        }
  \item {$S_{res}=\sum\limits_{\pi \in \Pi}{\chi(\pi)w_{\pi}}$}
\end{itemize}
  Построить маршрут максимизирующий $S_{res}$
\end{frame}

\begin{frame}
  \frametitle{Результат алгоритма}
\putImg<+->{pic06-lh.png}
\begin{itemize}
  \item $l_i$ --- проекция $i$-го галса на прямую $l$
  \item $h_i$ --- разница между галсом $i$ и $i+1$
  \item $S_{res}$ --- доля собранных частиц от начального распределения
\end{itemize}

\end{frame}

\section{Симуляция эволюции распределения} % Since this is the start of a new section, our recurring outline will appear here.
\begin{frame}
  \frametitle{Сервисы симулятора}
\begin{columns}
\column{.6\textwidth}
\putImg<+>{pic08-1.png}
\putImg<+>{pic08-2.png}
\column{.4\textwidth}
\begin{itemize}
\item<.(-1)->{демонстрация распределения в каждый момент прохождения маршрута}
\item<.>{Статистика
\begin{itemize}
  \item{\color{red} прогресс поиска}
  \item{\color{blue} поисковая производительность}
\end{itemize}
}
\end{itemize}
\end{columns}
\end{frame}

\begin{frame} 
\frametitle{Примеры моделей изменения распределения}
\begin{itemize}
\item{случайное блуждание с произвольным $\Delta t$ в качестве шага, $v \in [0, vMax]$}
\item{направленное движение в одном из фиксированных направлений}
\item{притяжение-отталкивание от фиксированных точек плоскости}
\end{itemize}
\end{frame}

\begin{comment}
\begin{frame}
\frametitle{Проблемa}
\begin{itemize}
  \item{ядро нужно применять раз в $\Delta t$ из физических соображений}
  \item{$\frac{1}{\Delta t} \ll 60FPS$} 
\end{itemize}
\end{frame}

\begin{frame}
\frametitle{Решение проблемы}
\putImg<+->{keyframe.png}
\begin{itemize}
  \item{приближенные версии частично примененных ядер $K_{\frac{1}{3}}, K_{\frac{2}{3}}$}
  \item{погрешность не накапливается, так как минимум предыдущий кадр ключевой}
\end{itemize}
\end{frame}
\end{comment}

\begin{comment}
\begin{frame}
  \frametitle{Проблема}
\begin{columns}
\column{.6\textwidth}
\putImg<+>{pic07-1.png}
\putImg<+>{pic07-2.png}
\column{.4\textwidth}
\only<.>{изначально выделенной текстуры недостаточно}
\end{columns}
\end{frame}

\begin{frame}
  \frametitle{Решение проблемы}
\begin{columns}
\column{.6\textwidth}
\putImg<+>{pic07-3.png}
\putImg<+>{pic07-4.png}
\putImg<+>{pic07-5.png}
\column{.4\textwidth}
\only<.(-2)->{Начинает выходить?}
\begin{itemize}
  \item<.(-1)->{перецентрировать}
  \item<.(0)->{увеличить x2 (совместить четыре ячейки в одну)}
\end{itemize}
\end{columns}
\end{frame}
\end{comment}

\section{Алгоритм построения маршрута}
\begin{frame}
\frametitle{Статический алгоритм}
\begin{itemize}
\item{$dp[cntHor][row][col][move][last]$ --- максимальное суммарный вес частиц, который можно собрать}
\item{$row, col$ --- текущий строка и столбец в которой находится средство поиска}
\item{$cntHor$ --- количество горизонтальных ходов}
\item{$last$ --- количество строк без галсирования}
\item{$move$ --- тип последнего хода из $\{L, LU, R, RU\}$} 

\end{itemize}
\end{frame}

\begin{frame}
\frametitle{Статический алгоритм: переходы}
\begin{itemize}
\item{$(cntHor, row, col) \to (cntHor, row+1, col)/(cntHor+1,row, col \pm 1)$}
\\
\item{$((cntHor_i, row_i, col_i, curMove, last_i), nextMove) \to 
(cntHor_{i+1}, row_{i+1}, col_{i+1}, nextMove, last_{i+1})$}
\end{itemize}
\end{frame}

\begin{frame}
  \frametitle{Статический алгоритм: переходы}
\begin{columns}
\column{.4\textwidth}
\begin{itemize}
\item{$L \to L, LU$}
\item{$R \to R, RU$}
\item{$LU \to LU, L, R$}
\item{$RU \to RU, L, R$}
\end{itemize}
\column{.4\textwidth}
\putImg<+>{dpzones}
\end{columns}
\end{frame}

\begin{frame}
\frametitle{Статический алгоритм: порядки величин}
\begin{itemize}
\item{$row \approx 200$}
\item{$col \approx 50$}
\item{$cntHor \approx 10^3$}
\item{$last \approx 16$ --- более дальние мало влияют}
\item{$row \cdot col \cdot cntHor \cdot last \cdot 4 \approx 6.4 \cdot 10^8$}
\end{itemize}
\end{frame}

\begin{frame}
\frametitle{Результаты работы статического алгоритма}
\begin{columns}
\column{.6\textwidth}
\putImg<+>{pic01-clear.png}
\putImg<+>{pic01-1h.png}
\putImg<+>{pic01-2h.png}
\putImg<+>{pic01-3h.png}
\putImg<+>{pic01-4h.png}
\putImg<+>{pic01-8h.png}
\putImg<+->{pic01-16h.png}

\column{.4\textwidth}
\begin{itemize}
\setbeamercovered{dynamic}
\item<.(-6)>{исходное распределение}
\item<.(-5)>{1 час}
\item<.(-4)>{2 часa}
\item<.(-3)>{3 часa}
\item<.(-2)>{4 часa}
\item<.(-1)>{8 часов}
\item<.(0)>{16 часов}
\end{itemize}
\end{columns}
\end{frame}

\begin{frame}
\frametitle{Корректировка построенного пути}
\begin{columns}
\column{.6\textwidth}
\putImg<+>{pic02-init.png}
\putImg<+>{pic02-before.png}
\putImg<+>{pic02-after.png}

\column{.4\textwidth}
\begin{itemize}
\setbeamercovered{dynamic}
\item<.(-2)>{изначально построенный путь}
\item<.(-1)>{со временем путь устарел}
\item<.(0)>{перестроим путь}
\end{itemize}
\end{columns}

\end{frame}
\begin{frame}
\frametitle{Корректировка построенного пути}
\begin{itemize}
\item{$rest$ --- непройденная часть построенного пути}
\item{$S'_{rest}$ --- планировалось собрать на симуляторе, когда строили путь (без учета диффузии)}
\item{$S''_{rest}$ --- планируется на симуляторе к текущему моменту (без учета диффузии)}
\item{$S''_{rest} \le k \cdot S'_{rest}$ --- перестроить маршрут с текущей точки на оставшееся время}
\item{$k \approx 0.98$}
\end{itemize}
\end{frame}

\begin{comment}
\begin{frame}
\frametitle{Локальная оптимизация пути}
\begin{itemize}
\item{$k$ --- дискретных значений $h_i$}
\item{необходимо осуществить более точную подстройку $h_i$}
\item{$l_i$ --- фиксированы}
\item{$h_j, (j \neq i) \land (j \neq i+1)$ --- фиксированы, локально изменяем $\frac{h_i}{h_{i+1}}$ пока можно улучшить $S_{res}$}
\end{itemize}
\end{frame}
\end{comment}

\section{Полученные результаты}
\begin{frame}
  \frametitle{Сравнение}
\begin{table}[ht]
  \centering
  \begin{tabular}{|l|l|l|l|l|}
\hline
    № & $T_{old}$ & $R_{old}$ & Сравнение по & Сравнение по \\
         & & & результату $R$ & времени $T$ \\
\hline
    1 & 3.6 & 87.5\% & 94.7\% & 2.9 \\ 
    2 & 3.6 & 83.3\% & 92.6\% & 1.9 \\
    3 & 3.6 & 83.3\% & 86.1\% (4.5) & 3.6 \\
    4 & 3.6 & 100.0\%& 100.0\%& 3.3 (99.7\%) \\
\hline

  \end{tabular}
\end{table}

\end{frame}

\begin{frame}
\frametitle{Полученные результаты}
\begin{itemize}
\item{Реализован инструмент, рассчитывающий изменение распределения частиц с учетом поискового средства в реальном времени. Инструмент используется для визуализации и оценки эффективности алгоритмов поиска}
\item{Разработан и реализован алгоритм построения пути поиска методом ``Параллельное галсирование'', обеспечивающий нахождение объекта с вероятностью $\ge 90\%$ в большинстве случаев, за приемлемое время поиска}
\end{itemize}
\end{frame}

\begin{frame}
  \begin{center}
    \Huge
    {\color{blue} Вопросы?}
  \end{center}
\end{frame}

\end{document}

%%% Local Variables:
%%% mode: latex
%%% TeX-master: t
%%% End:
