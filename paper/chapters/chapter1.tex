%-*-coding: utf-8-*-
\chapter{Постановка задачи}
\label{chapSVD}

\section{Задача построения маршрута поиска в oбщeм случае}


\subsection{Расширения задачи коммивояжера}
В классической формулировке задача коммивояжера(traveling salesman problem, TSP) звучит так:
 Дан взвешенный граф необходимо найти цикл, минимального веса, посещающий все его вершины.
 Евклидовым TSP --- называется частный случай TSP, когда весами ребер являются
расстояния на плоскости. 
 Задача построения маршрута поиска объекта на первый взгляд очень похожа на задачу коммивояжера.
Однако в реальности особенности задачи оказываются существенными:
\begin{itemize}
\item{Средство поиска в один момент времени посещает несколько вершин, а именно все вершины
попадающие в круг определенного радиуса с центром в его текущем положении.
Существует расширение TSP под названием GTSP или обобщенная задача коммивояжера, которое решает
следующую задачу:
Дан взвешенный граф и разбиение его вершин на \textbf{непересекающиеся} множества, необходимо
найти цикл минимального веса, посещающий хотя бы одну вершину из каждого множества.
Существует сведение данной задачи к TSP не увеличивающее размерности и доказывающее ее $NP$-полноту.
Но к сожалению данное расширение неприменимо к нашей задаче, так как множества могут пересекаться.}
\item{Зачастую все вершины посетить физически невозможно. 
Соответственно выделяются вершины в которых более вероятно обнаружить объект.
 Сопоставим вершине $v$ величину $p_v$ --- вероятность обнаружить объект в этой вершине.
 $p_{path}=\sum\limits_{v\in path}p_v$. На практике длина путей с $p_{path} \ge 0.99$ может превышать
длину путей с $p_{path} \ge 0.9$ в десятки раз. То есть длина пути растет экспоненциально
в зависимости от $p_{path}$. Следовательно необходимым параметром задачи становится максимальная длина
пути (или время поиска с физической точки зрения). Известно обобщение Profit Based TSP:
каждая вершина имеет сопоставленное ей значение приза $p_v$, при посещении вершины к нашему
счету добавляется $p_v-t_v$, где $t_v$---время посещения, необходимо составить маршрут с
наибольшей суммой призов. К сожалению наша задача и здесь сравнительно более общая, так как
величины призов могут изменяться нелинейно.}
\item{Распределение $p_v$ действительно может быть не статично по времени и изменяться согласно
заданной модели. Следовательно время и расстояние не взаимозаменяемы в поставленной задаче и
оптимальное значение $p_{path}$ может быть различно если мы фиксируем один из параметров.
В поставленной задаче будет фиксированно время. Однако все модели изменения обладают свойством:
распределение изменяется непрерывно, перераспределяясь не превышая фиксированную скорость,
не появляется извне и не исчезает (будем считать что при посещенеии призы ``собираются'' и исчезают).
Соответственно $p_{v,t}=\sum_{pos(q_i)=t}q_i$ и приз $q_i$ в вершине $v$ мы можем собрать лишь в какие-то
промежутки времени. Обобщение Time Windows TSP решают соответсвующую задачу:
вершину $v$ --- можно посетить лишь во время $[t_{v,l}; t_{v,r}]$. Проблема сведения к этому обобщению
в том, что значительно увеличивается количество вершин, в частности новая вершина будет
сопоставлена $j$-му моменту, когда приз $q_i$ оказался в вершине $v$.} 
Kinetic TSP (коммивояжер с движущимися целями), Profit Based TSP ( }
\item{}
\end{itemize}

\subsection{Формулировка задачи построения маршрута поиска}

\FloatBarrier
%%% Local Variables:
%%% mode: latex
%%% TeX-master: t
%%% End:
