\chapter{Сравнение с существующими решениями}
\section{Обзор существующих решений}
Был разработан алгоритм строящий маршруты поиска, которые удовлетворяют фиксированным паттернам
стратегии ``Параллельное галсирование''. Задача построения маршрутов имеющих определенную
структуру достаточно специфична. К сожалению, методы, решающие поставленную задачу (или
ее частные случаи) не были найдены в открытых источниках.

 Однако существует комплекс инструментов, ранее разработанный в ``Кронштадт Технологии'',
который предоставляет возможность строить маршруты разными стратегиями поиска, несколькими
поисковыми средствами, оптимизировать параметры совместного поиска и некоторые другие возможности.
Рассмотренный в данной работе метод должен прийти на замену существующему в данном комплексе,
поэтому в первую очередь должно быть осуществлено сравнение с этим методом.

Далее будет произведено сравнение с существующим алгоритмом построения маршрутов стратегией
``Параллельное галсирование''.

\FloatBarrier
\section{Сравнение со старым алгоритмом построения маршрутов ``Кронштадт Технологии''}
Старый алгоритм имеет немного другой интерфейс. На вход принимается район в котором необходимо
осуществить галсирование. Единственная поддерживаемая модель изменения распределения ---
случайное блуждание с определенной интенсивностью $I$. Интенсивность $I$ определяет сколько
времени потребуется частицам, чтобы не менее $p\%$ отдалились на единицу длины. Исходя из
интенсивности рассчитывается расстояние между галсами --- одинаковое на всем протяжении маршрута.
Причем допускается пропуск частиц суммарного веса $\epsilon$ на каждом галсе.

Сравнение будет происходить следующим образом. Тест 1 продемонстрирует несостоятельность старого
метода в условии другой модели изменения распределения. Тест 2 покажет преимущества нового метода
при неравномерных распределениях. Все последующие тесты будут иметь равномерное распределение
и модель изменения --- случайное блуждание. Тест 3 покажет важность корректировки маршрута со
временем, так как район расширяется и ширина галса со временем должны быть увеличена.
Тест 4 покажет способность нового метода учитывать досмотренные зоны во время движения вверх.
Тест 5 проверит способность нового метода подстраивать ширину галса в простейшем случае, когда
частицы не передвигаются.

В старом методе время поиска не фиксируется, а возвращается по результату работы алгоритма.
Таким образом для каждого теста новым алгоритм будет построено два маршрута. Первый маршрут будет
занимать минимальное время и показывать результат не хуже старого алгоритма. Прохождение второго
маршрута будет занимать ровно столько же времени и показывать какой вес сможет собрать новый
алгоритм.

\subsection{Teст 1}
Воспользуемся моделью изменения распределения --- притяжение к определенным точкам. Пусть
частицы стремяться приблизиться к берегу, расположенному слева от района поиска. Новый алгоритм
скорректирует маршрут, в то время как старый продолжит обследовать область их начально
местоположения.


\subsection{Teст 2}
\subsection{Teст 3}
\subsection{Teст 4}
\subsection{Teст 5}
\FloatBarrier
