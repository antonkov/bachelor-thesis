\chapter{Сравнение с существующими решениями}
\section{Обзор существующих решений}
Был разработан алгоритм строящий маршруты поиска, которые удовлетворяют фиксированным паттернам
стратегии ``Параллельное галсирование''. Задача построения маршрутов имеющих определенную
структуру достаточно специфична. К сожалению, методы, решающие поставленную задачу (или
ее частные случаи) не были найдены в открытых источниках.

 Однако существует комплекс инструментов, ранее разработанный в ``Кронштадт Технологии'',
который предоставляет возможность строить маршруты разными стратегиями поиска, несколькими
поисковыми средствами, оптимизировать параметры совместного поиска и некоторые другие возможности.
Рассмотренный в данной работе метод должен прийти на замену существующему в данном комплексе,
поэтому в первую очередь должно быть осуществлено сравнение с этим методом.

Далее будет произведено сравнение с существующим алгоритмом построения маршрутов стратегией
``Параллельное галсирование''.

\FloatBarrier
\section{Сравнение со старым алгоритмом построения маршрутов ``Кронштадт Технологии''}
Старый алгоритм имеет немного другой интерфейс. На вход принимается район в котором необходимо
осуществить галсирование. Единственная поддерживаемая модель изменения распределения ---
случайное блуждание с определенной интенсивностью $I$. Интенсивность $I$ определяет сколько
времени потребуется частицам, чтобы не менее $p\%$ отдалились на единицу длины. Исходя из
интенсивности рассчитывается расстояние между галсами --- одинаковое на всем протяжении маршрута.
Причем допускается пропуск частиц суммарного веса $\epsilon$ на каждом галсе.

Сравнение будет происходить следующим образом. Тест 1 продемонстрирует несостоятельность старого
метода в условии другой модели изменения распределения. Тест 2 покажет преимущества нового метода
при неравномерных распределениях. Все последующие тесты будут иметь равномерное распределение
и модель изменения --- случайное блуждание. Тест 3 покажет важность корректировки маршрута со
временем, так как район расширяется и ширина галса со временем должны быть увеличена.
Тест 4 покажет способность нового метода учитывать досмотренные зоны во время движения вверх.
Тест 5 проверит способность нового метода подстраивать ширину галса в простейшем случае, когда
частицы не передвигаются.

В старом методе время поиска не фиксируется, а возвращается по результату работы алгоритма.
Таким образом для каждого теста новым алгоритм будет построено два маршрута. Первый маршрут будет
занимать минимальное время и показывать результат не хуже старого алгоритма. Прохождение второго
маршрута будет занимать ровно столько же времени и показывать какой вес сможет собрать новый
алгоритм.

Для каждого маршрута будет показана статистика --- прогресс и поисковая производительность
на протяжении всего времени поиска. Синий график поисковой производительности имеет
относительный масштаб и может быть использован только для сравнения поисковой производительности
в рамках одного поиска (если его абсолютная величина на одном из двух графиков меньше --- это
не значит, что поисковая производительность в этом запуске хуже чем во втором).
\subsection{Teст 1}
Воспользуемся моделью изменения распределения --- притяжение к определенным точкам. Пусть
частицы стремяться приблизиться к берегу, расположенному слева от района поиска. Новый алгоритм
скорректирует маршрут, в то время как старый продолжит обследовать область их начально
местоположения.

 Старый алгоритм запросил 3.6 часа и собрал 85.7\% суммарного веса.
За это же время новый алгоритм собрал 94.7\%, а для сбора 85.7\% ему потребовалось
менее 2.95 часа.

\subsection{Teст 2}

\subsection{Teст 3}
\subsection{Teст 4}
\subsection{Teст 5}
\begin{figure}[ht]
  \begin{center}
    \subfigure[Старый алгоритм]{%
        \putImgx{0.4\textwidth}{t1op}
        \label{t1op}
    }
    \subfigure[Статистика]{%
        \putImgx{0.4\textwidth}{t1os}
        \label{t1os}
    } \\
    \subfigure[Новый за это время]{%
        \putImgx{0.4\textwidth}{t1np}
        \label{t1np}
    }
    \subfigure[Статистика]{%
        \putImgx{0.4\textwidth}{t1ns}
        \label{t1ns}
    } \\
    \subfigure[Новый с такой же вероятностью]{%
        \putImgx{0.4\textwidth}{t1sp}
        \label{t1sp}
    } 
    \subfigure[Статистика]{%
        \putImgx{0.4\textwidth}{t1ss}
        \label{t1ss}
    } 
\end{center}
  \caption{Тест 1. Модель изменения --- приближение к прямой слева}
\end{figure}

\FloatBarrier

\section{Результаты}

\begin{table}[ht]
  \centering
  \begin{tabular}{|l|l|l|l|l|}
\hline
    Номер & Время старого & Результат старого & Результат & Время с тем \\
    теста     &&& с тем же временем & же результатом \\
\hline
    1 & 3.6 & 87.5\% & 94.7\% & 2.95 \\ 
\hline

  \end{tabular}
  \captionsetup{justification=centering}
  \caption{Сводная таблица результатов работы на тестах}
  \label{results}
\end{table}

\FloatBarrier