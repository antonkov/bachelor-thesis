%-*-coding: utf-8-*-

\chapter{Симуляция эволюции распределения}
\section{Примеры моделей и распределений}
\begin{figure}[ht]
  \begin{center}
    \subfigure[Композиция стандартных распределений для задания начального распределения]{%
        \putImgx{0.4\textwidth}{pic03-1}
        \label{simul:init}
    }
    \subfigure[Изменения распределения согласно модели случайного блуждания]{%
        \putImgx{0.4\textwidth}{pic03-4}
        \label{simul:after}
    }
  \end{center}
  \caption{Пример симуляции изменения распределения без маршрута}
\end{figure}

\FloatBarrier
\subsection{Разновидности начальных распределений}
Начальное распределение частиц фактически может быть представлено любой двумерной функцией,
интеграл которой по всей плоскости равен единице. Однако для большинства применений кажется
достаточным задание начального распределения как композиции стандартных двумерных распределений
(рис. \ref{simul:init}), таких как равномерное распределение в произвольной области или
 нормальное распределение задаваемое еллипсом, содержащим $3\sigma$ вероятности и тому подобное.
 Инструмент симуляции предоставляет удобный инструмент их задания.
\FloatBarrier
\subsection{Разновидности моделей изменения распределения}
Простейший пример модели изменения распределения --- модель случайных блужданий.
(рис. \ref{simul:after})
\FloatBarrier
