% -*-coding: utf-8-*-
\startprefacepage


 Существующие подходы строят весь маршрут исходя из статичных данных в начальный момент
времени --- информации о начальном распределении и модели его распределения.
Таким образом главным недостатком существующих подходов является необходимость
разработки нового алгоритма для всех различных моделей изменения распределения.
Учитывая тот факт, что подобрать правильную модель, хорошо приближающую реальность,
 крайне непросто, возникает необходимость разработки алгоритма,
 работающего единообразно на широком классе различных моделей.
 Основная идея рассматриваемого подхода --- использование симулятора для получения
 информации о распределении в любой момент времени при построении маршрута.
 Таким образом при планирование пользователь в первую очередь выберет модель,
 как можно лучше приближающую реальность в данном случае, а после запустит единственный алгоритм.

Концепт данной задачи был продемонстрирован на одной из выставок. К задаче был проявлен 
интерес и было решено внедрить ее в комплекс расчетных морских задач.

В главе 1 решаемая задача будет рассмотрена более подробно.
 Описаны классы маршрутов, получаемые при использовании стратегии ``Параллельное галсирование''. 
Будут приведены особенности задачи, которые отличают ее от классической задачи коммивояжера
 и делают невозможным использование ранее разработанных методов решения TSP
 для решения исходной задачи в общем случае.

В главе 2 будут рассмотрены вопросы, связанные с разработкой симулятора на CUDA.
 Обозначены предоставляемые им сервисы.

В главе 3 будет описан алгоритм построения маршрутов согласно стратегии ``Параллельное галсирование''.

\FloatBarrier
