%-*-coding: utf-8-*-
\chapter{Алгоритм построения маршрутов согласно стратегии
``Параллельное галсирование''}
\section{Алгоритм построения маршрута при
 фиксированном распределении}

\def\dprule{ \rule[-2ex]{0pt}{4ex} }

Для построения маршрута в случае сохранения частицами их положений будет использован метод
динамического программирования. Далее будут описаны состояния алгоритма, его переходы и
вспомогательные значения посчитанные на матрице распределения.

Прежде всего введем сетку на прямоугольнике на котором рассматривается распределение.
Пусть размер сетки будет $H \times W$, размеры ячеек будут одинаковыми, но не обязаны
быть квадратными или выровненными по пикселям текстуры. В частности мы будем стремиться
к очень высокому разрешению по строчкам для более точного определения расстояния между
соседними галсами и нам будет достаточно небольшого разрешения по столбцам. В частности
если размеры текстуры $T_H \times T_W$, а размеры прямоугольника в мире $R_H \times R_W$,
то мы хотим иметь размеры сетки $H = T_H$ и $W \approx \frac {R_w} {1000}$.

\subsection{Состояния алгоритма}
Значением динамического программирования $dp(state)$ для определенного состояния будет
суммарный вес собранных частиц к моменту времени определяемому состоянием.

Состоянием динамического программирования является кортеж $(cntHor, row, col, move, notCleared)$.
\begin{itemize}
\item $(row, col)$ --- строка и столбец ячейки в которой закончен маршрут к текущему моменту времени
соответственно.
\item $cntHor$ --- количество горизонтальных перемещений (между столбцами). Если обозначить
за $time_W$ --- время перемещения между двумя соседними ячейками по горизонтали, а за $time_H$ ---
по вертикали, то текущее время можно посчитать как $curTime = time_H \cdot row + time_w \cdot cntHor$.
Здесь использованы предположения, что мы никогда не возвращаемся назад по строчками (исходя из
выбранной стратегии поиска), всегда двигаемся с максимальной скоростью и в начальный момент
времени мы находимся в нулевой строчке.

\end{itemize}

\subsection{Вспомогательные матрицы с частичными суммами весов}
\subsection{Переходы между состояниями}
\subsection{Оценка времени работы}
\begin{table}[ht]
  \centering
\begin{tabular}{|l|l|l|l|}
  \hline
  Текущий & Следующий & Собрано на текущем ходу & Новое состояние  \\
  ход & ход & & \\
\hline 
\dprule $L$&$L$&$v[row][col][notCleared]$&$dp[cntHor+1][row][col-1][L][notCleared]$\\
\hline 
\dprule $L$&$LU$&$v[row][col][notCleared]$&$dp[cntHor][row+1][col][LU][0]$\\
\dprule  && $cs[row][col][LDC]$ & \\
\hline 
\dprule $R$&$R$&$v[row][col][notCleared]$&$dp[cntHor+1][row][col+1][R][notCleared]$\\
\hline 
\dprule $R$&$RU$&$v[row][col][notCleared]$&$dp[cntHor][row+1][col][RU][0]$\\
\dprule  && $cs[row][col][RDC]$ & \\
\hline 
\dprule $LU$&$LU$&$v[row][col][0]$&$dp[cntHor][row+1][col][LU][notCleared+1]$\\
\dprule && $hr[row][col]$ & \\
\hline 
\dprule $LU$&$L$&$v[row][col][0]$&$dp[cntHor+1][row][col-1][L][notCleared]$\\
\dprule && $cs[row][col][RUC]$ & \\
\hline 
\dprule $LU$&$R$&$v[row][col][0]$&$dp[cntHor+1][row][col+1][R][notCleared]$\\
\dprule  && $cs[row][col][LUC]$ & \\
\hline 
\dprule $RU$&$RU$&$v[row][col][0]$&$dp[cntHor][row+1][col][RU][notCleared+1]$\\
\dprule && $hl[row][col]$ & \\
\hline 
\dprule $RU$&$L$&$v[row][col][0]$&$dp[cntHor+1][row][col-1][L][notCleared]$\\
\dprule && $cs[row][col][RUC]$ & \\
\hline 
\dprule $RU$&$R$&$v[row][col][0]$&$dp[cntHor+1][row][col+1][R][notCleared]$\\
\dprule && $cs[row][col][LUC]$ & \\
\hline 
\end{tabular}
\captionsetup{justification=centering}
\caption{Переходы из состояния $dp[cntHor][row][col][curMove][notCleared]$}
\label{table:dp}
\end{table}
\FloatBarrier

\section{Корректировка маршрута}
\FloatBarrier
