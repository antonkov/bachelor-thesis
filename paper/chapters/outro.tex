\startconclusionpage

В данной работе был разработан алгоритм, построения маршрутов поиска стратегией
``Параллельное галсирование''. Алгоритм позволяет работать с неравномерными
распределениями и разлчиными моделями изменения распределений со временем.

В качестве недостатков алгоритма, которые подлежат дальнейшему улучшению,
нужно отметить большое время работы при максимальном разрешении сетки
по строкам $T_H$. Хотя такое разрешение необходимо для наиболее точной
подстройки расстояния между галсами, на практике удается
обойтись меньшим разрешением без значительных потерь в результативности, при
этом время работы снижается до приемлимого уровня.

Для целей визуализации и в качестве вспомогательного средства для корректировки
маршрутов со временем был разработан симулятор прохождения маршрутов.
Симулятор основан на технологии CUDA, таким образом при высоком разрешением
поля симуляции удается поддерживать достаточную скорость работы для
визуализации прохождения маршрута в ускоренном режиме.

В качестве дальнейших улучшений симулятора можно рассмотреть повышение
разрешения, посредством более продвинутых техник осуществления свертки
изображения (Image Convolution) на GPU.

Разработанный алгоритм должен заменить существующий алгоритм
``Кронштадт технологии'', сравнение с которым было осуществлено
в главе 4. Было показано, что разработанный алгоритм не уступает старому и
значительно превосходит его в случае неравномерных распределений и
моделей их изменения отличных от случайных блужданий. 

\FloatBarrier
