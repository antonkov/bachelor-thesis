% -*-coding: utf-8-*-
% This is an AMS-LaTeX v. 1.2 File.

\documentclass{report}


%\usepackage{pscyr}
%\renewcommand{\rmdefault}{fjn}
%\renewcommand{\ttdefault}{fcr}

%\usepackage{showkeys}
\usepackage[T2A]{fontenc}
\usepackage[utf8x]{inputenc}
\usepackage[english,russian]{babel}
\usepackage{expdlist}
\usepackage[pdftex]{graphicx}
\usepackage{amsmath}
\usepackage{amssymb}
\usepackage{amsthm}
\usepackage{amsfonts}
\usepackage{amsxtra} 
\usepackage{wrapfig}
\usepackage{sty/dbl12}
\usepackage{srcltx}
\usepackage{epsfig}
\usepackage{verbatim}
\usepackage{sty/rac}
%\usepackage[russian]{sty/ralg}
\usepackage{listings}
\usepackage{placeins}
\usepackage{caption}
\usepackage{subfigure}
%\usepackage{floatrow}
\captionsetup[table]{position=t,justification=raggedright,slc=off}

%\usepackage[
%    top    = 2.00cm,
%    bottom = 2.00cm,
%    left   = 3.00cm,
%    right  = 1.50cm]{geometry}
\hoffset = -10mm
\voffset = -20mm
\textheight = 230mm
\textwidth = 165mm

%%%%%%%%%%%%%%%%%%%%%%%%%%%%%%%%%%%%%%%%%%%%%%%%%%%%%%%%%%%%%%%%%%%%%%%%%%%%%%

% Redefine margins and other page formatting

%\setlength{\oddsidemargin}{0.5in}

% Various theorem environments. All of the following have the same numbering
% system as theorem.

\theoremstyle{plain}
\newtheorem{theorem}{Теорема}
\newtheorem{prop}[theorem]{Утверждение}
\newtheorem{corollary}[theorem]{Следствие}
\newtheorem{lemma}[theorem]{Лемма}
\newtheorem{question}[theorem]{Вопрос}
\newtheorem{conjecture}[theorem]{Гипотеза}
\newtheorem{assumption}[theorem]{Предположение}

\theoremstyle{definition}
\newtheorem{definition}[theorem]{Определение}
\newtheorem{notation}[theorem]{Обозначение}
\newtheorem{condition}[theorem]{Условие}
\newtheorem{example}[theorem]{Пример}
\newtheorem{algorithm}[theorem]{Алгоритм}
%\newtheorem{introduction}[theorem]{Introduction}

\renewcommand{\proof}{\\\textbf{Доказательство.}~}

%\def\startprog{\begin{lstlisting}[language=Java,basicstyle=\normalsize\ttfamily]}

%\theoremstyle{remark}
%\newtheorem{remark}[theorem]{Remark}
%\include{header}
%%%%%%%%%%%%%%%%%%%%%%%%%%%%%%%%%%%%%%%%%%%%%%%%%%%%%%%%%%%%%%%%%%%%%%%%%%%%%%%

\numberwithin{theorem}{chapter}        % Numbers theorems "x.y" where x
                                        % is the section number, y is the
                                        % theorem number

%\renewcommand{\thetheorem}{\arabic{chapter}.\arabic{theorem}}

%\makeatletter                          % This sequence of commands will
%\let\c@equation\c@theorem              % incorporate equation numbering
%\makeatother                           % into the theorem numbering scheme

%\renewcommand{\theenumi}{(\roman{enumi})}

%%%%%%%%%%%%%%%%%%%%%%%%%%%%%%%%%%%%%%%%%%%%%%%%%%%%%%%%%%%%%%%%%%%%%%%%%%%%%%


%%%%%%%%%%%%%%%%%%%%%%%%%%%%%%%%%%%%%%%%%%%%%%%%%%%%%%%%%%%%%%%%%%%%%%%%%%%%%%%

%This command creates a box marked ``To Do'' around text.
%To use type \todo{  insert text here  }.

\newcommand{\todo}[1]{\vspace{5 mm}\par \noindent
\marginpar{\textsc{ToDo}}
\framebox{\begin{minipage}[c]{0.95 \textwidth}
\tt #1 \end{minipage}}\vspace{5 mm}\par}

%%%%%%%%%%%%%%%%%%%%%%%%%%%%%%%%%%%%%%%%%%%%%%%%%%%%%%%%%%%%%%%%%%%%%%%%%%%%%%%

\binoppenalty=10000
\relpenalty=10000

\begin{document}

% Begin the front matter as required by Rackham dissertation guidelines

\initializefrontsections

\pagestyle{title}

\begin{center}
Санкт-Петербургский национальный исследовательский университет \\ информационных технологий, механики и оптики

\vspace{2cm}

Кафедра компьютерных технологий

\vspace{3cm}

{\Large А. П. Ковшаров}

\vspace{2cm}

\vbox{\LARGE\bfseries
Оптимизация параметров стратегий поиска объектов на море}

\vspace{4cm}

Бакалаврская работа 

\vspace{1cm}

{\Large Научный руководитель: А. С. Ковалев}

\vspace{5cm}

Санкт-Петербург\\ 2015
\end{center}

\newpage

\setcounter{page}{3}
\pagestyle{plain}

%\dedicationpage{Put a dedication here}
% Dedication page

%\startacknowledgementspage
% Acknowledgements page
%{Put Acknowledgements here}

% Table of contents, list of figures, etc.
\tableofcontents
%\listoffigures


\def\t#1{\mbox{\texttt{\hbox{#1}}}}
\def\b#1{\textbf{#1}}
\def\tb#1{\t{\b{#1}}}

\def\cln#1{\t{#1}}
\def\pcn#1{\t{#1}}
\newcommand{\p}{\par Здесь будет текст...}
\def\putImgx#1#2{
  \includegraphics[width=#1]{../img/#2}
}

\def\putImg#1{
  \includegraphics{../img/#1}
}

\def\drawfigure#1#2{
        \begin{figure}
        \putImg{#1}
        \caption{#2}
        \label{#1}
        \end{figure}
}
\def\drawfigurex#1#2#3#4{
        \begin{figure}[ht]
          \begin{center}
            \putImgx{#4}{#1}
            \caption{#2}
            \label{#3}
          \end{center}
        \end{figure}
}
 
% Chapters
\startthechapters
% -*-coding: utf-8-*-
\startprefacepage


 Существующие подходы строят весь маршрут исходя из статичных данных в начальный момент
времени --- информации о начальном распределении и модели его распределения.
Таким образом главным недостатком существующих подходов является необходимость
разработки нового алгоритма для всех различных моделей изменения распределения.
Учитывая тот факт, что подобрать правильную модель, хорошо приближающую реальность,
 крайне непросто, возникает необходимость разработки алгоритма,
 работающего единообразно на широком классе различных моделей.
 Основная идея рассматриваемого подхода --- использование симулятора для получения
 информации о распределении в любой момент времени при построении маршрута.
 Таким образом при планирование пользователь в первую очередь выберет модель,
 как можно лучше приближающую реальность в данном случае, а после запустит единственный алгоритм.

Концепт данной задачи был продемонстрирован на одной из выставок. К задаче был проявлен 
интерес и было решено внедрить ее в комплекс расчетных морских задач.

В главе 1 решаемая задача будет рассмотрена более подробно.
 Описаны классы маршрутов, получаемые при использовании стратегии ``Параллельное галсирование''. 
Будут приведены особенности задачи, которые отличают ее от классической задачи коммивояжера
 и делают невозможным использование ранее разработанных методов решения TSP
 для решения исходной задачи в общем случае.

В главе 2 будут рассмотрены вопросы, связанные с разработкой симулятора на CUDA. Обозначены предоставляемые им сервисы.

В главе 3 будет описан алгоритм построения маршрутов согласно стратегии ``Параллельное галсирование''.

\begin{comment}
Задача определения кратчайшего расстояния до конфигурации отрезков
(dist2segments), являясь частным случаем задачи dist2sites -- задачи о
минимальном расстоянии до произвольного набора линейных данных,
является одной из базовых задач вычислительной геометрии (computational
geometry) \cite{PrSh}. На решении данной задачи базируются решения некоторых задач
об избежании столкновений (collision avoidance problem) \cite{MarNav}, некоторых задач
из области геоинформационных систем (Geographic information system, GIS) \cite{CGinGIS},
текстурирования рельефа, математического моделирования движения твердых
тел в жидкости и многих других.

Задача определения кратчайшего расстояния до конфигурации
отрезков имеет очевидное решение -- это полный
перебор с отсечением, имеющий линейную сложность ($O(n)$) \cite{DnCG}. Однако, если
множество точек-запросов $Q$ имеет достаточно большую мощность, обычно
рассматривают так называемую диаграмму Вороного для отрезков (segment
Voronoi diagram, SVD) \cite{PrSh, CGAL}. Эта структура данных похожа на диаграмму
Вороного для точек (point Voronoi diagram) \cite{PrSh, CGAL}, однако она не может быть
представлена реберным списком с двойными связями (double-connected edge
list, DCEL, РСДС) \cite{PrSh, CGAL} в силу своей нелинейности. Хотя, конечно, можно
создать приближенный РСДС, сколь угодно точно описывающий диаграмму
Вороного для отрезков.

Очень часто возникает необходимость в достаточно
больших количествах запросов на поиск ближайших отрезков.
Например, такая необходимость возникает при расчете физики движения судов в морских
тренажерах \cite{MarNav}. Судов может быть достаточно много, объектов, с которыми
необходимо рассчитать взаимодействие, тоже достаточно много. Это означает,
что запросы расстояний до ближайших отрезков идут достаточно часто, чтобы это стало проблемой
для ЭВМ, вычислительная мощность которых может быть недостаточно
большой для такого потока данных. Такого рода запросы называются
массовыми запросами, а такая задача -- массовой задачей.

Массовая задача в \cite{PrSh} определена следующим образом. Существует
фиксированный набор входных данных $S$. Требуется вычислить массовый
запрос $Q$, то есть ответить на некоторый поставленный вопрос для каждого
запроса из $Q$. Иногда такие задачи решаются в два этапа: предобработка 
(pre-processing) и вычисление запросов на некоторой структуре данных,
формирующейся на этапе предобработки и облегчающей поиск, что позволяет
сократить суммарное время по сравнению с последовательным решением
исходной задачи для каждого запроса.

Логично предположить, что такая структура данных должны быть
подобна хешу (hash) \cite{AHU} или дереву (tree) \cite{QT, SQT, FANN}. В работе \cite{NGRID} была предложена
такая структура данных -- многоуровневая сеть (n-grid). Если потребовать,
чтобы с каждой ячейкой сети ассоциировались только ближайшие к этой
ячейке отрезки, то задача сводится к перебору небольшого числа отрезков.

В данной работе будет описан новый подход к заполнению таких структур --
построение нижних огибающих функций расстояний до отрезков. Также будет описана
структура данных, основанная на заполнении квадродерева (quadtree) этим методом,
с доказательством того, что, при логарифмическом ($O(\log n)$) ограничении высоты дерева,
математическое ожидание числа перебираемых отрезков будет константой, и сравнением ее
с другими существующими реализациями.  
\end{comment}

\FloatBarrier

%-*-coding: utf-8-*-
\chapter{Постановка задачи}
\label{chapSVD}

\section{Задача построения маршрута поиска в oбщeм случае}


\subsection{Расширения задачи коммивояжера}
В классической формулировке задача коммивояжера(traveling salesman problem, TSP) звучит так:
 Дан взвешенный граф необходимо найти цикл, минимального веса, посещающий все его вершины.
 Евклидовым TSP --- называется частный случай TSP, когда весами ребер являются
расстояния на плоскости. 
 Задача построения маршрута поиска объекта на первый взгляд очень похожа на задачу коммивояжера.
Однако в реальности особенности задачи оказываются существенными:
\begin{itemize}
\item{Средство поиска в один момент времени посещает несколько вершин, а именно все вершины
попадающие в круг определенного радиуса с центром в его текущем положении.
Существует расширение TSP под названием GTSP или обобщенная задача коммивояжера, которое решает
следующую задачу:
Дан взвешенный граф и разбиение его вершин на \textbf{непересекающиеся} множества, необходимо
найти цикл минимального веса, посещающий хотя бы одну вершину из каждого множества.
Существует сведение данной задачи к TSP не увеличивающее размерности и доказывающее ее $NP$-полноту.
Но к сожалению данное расширение неприменимо к нашей задаче, так как множества могут пересекаться.}
\item{Зачастую все вершины посетить физически невозможно. 
Соответственно выделяются вершины в которых более вероятно обнаружить объект.
 Сопоставим вершине $v$ величину $p_v$ --- вероятность обнаружить объект в этой вершине.
 $p_{path}=\sum\limits_{v\in path}p_v$. На практике длина путей с $p_{path} \ge 0.99$ может превышать
длину путей с $p_{path} \ge 0.9$ в десятки раз. То есть длина пути растет экспоненциально
в зависимости от $p_{path}$. Следовательно необходимым параметром задачи становится максимальная длина
пути (или время поиска с физической точки зрения). Известно обобщение Profit Based TSP:
каждой вершине сопоставляется значение $p_v$, при посещении вершины к сумме призов 
добавляется $p_v-t_v$, где $t_v$---время посещения, необходимо составить маршрут с
наибольшей суммой призов. К сожалению наша задача и здесь сравнительно более общая, так как
величины призов могут изменяться нелинейно.}
\item{Распределение $p_v$ действительно может быть не статично по времени и изменяться согласно
заданной модели. Следовательно время и расстояние не взаимозаменяемы в поставленной задаче и
оптимальное значение $p_{path}$ может быть различно если мы фиксируем один из параметров.
В поставленной задаче будет фиксированно время. Однако все модели изменения обладают свойством:
распределение изменяется непрерывно, перераспределяясь не превышая фиксированную скорость,
не появляется извне и не исчезает (будем считать что при посещенеии призы ``собираются'' и исчезают).
Соответственно $p_{v,t}=\sum_{pos(q_i)=t}q_i$ и приз $q_i$ в вершине $v$ мы можем собрать лишь в какие-то
промежутки времени. Обобщение Time Windows TSP решают соответсвующую задачу:
вершину $v$ --- можно посетить лишь во время $[t_{v,l}; t_{v,r}]$. Проблема сведения к этому обобщению
в том, что значительно увеличивается количество вершин, в частности новая вершина будет
сопоставлена $j$-му моменту, когда приз $q_i$ оказался в вершине $v$.} 
\item{Другое обобщение для случая с движущимися вершинами --- Kinetic TSP или Moving Targets TSP.
Это классическое евклидово TSP, в котором вершины движутся в фиксированном направлении.
Однако ранее были рассмотрены только случаи где вершины движутся с фиксированной скоростью или
коммивояжер должны возвращаться в стартовую вершину после посещения каждой. В исходной
задаче скорость и направления перемещения призов могут изменяться. Кроме того сведение вновь
значительно увеличивает число вершин.}
\item{Фактически до текущего момента времени задача формулировалась в непрерывном случае.
При разработке алгоритма для ЭВМ необходимо провести некоторую дискретизацию.
Необходимое количество вершин для двумерной задачи растет квадратично и соответственно
приемлемое значение может достигать миллионов вершин. Такое количество вершин велико
даже для самых быстрых реализаций приближенных решений TSP.}
\item{В реальной жизни невозможно искать согласно произвольной траектории из-за технических
ограничений на передвижение поискового средства. Помимо этого недопустимo применение не
общепризнанных, зарекомендовавших себя стратегий поиска.}
\end{itemize}

\FloatBarrier
\subsection{Распределение частиц}
Использование понятия распределение вероятности обнаружения противника в дальнейшем будет
не совсем удобно, так как одна из основных частей задачи ``сбор'' вероятности --- не является
корректной операцией над вероятностями.

Введем альтернативное понятие --- распределение ``частиц''.
Частица $\pi : \Pi$ --- гипотеза изначального расположения объекта и его поведения в дальнейшем.
Вес частицы $w_{\pi}$ --- вероятность осуществления именно этой гипотезы.
$pos(\pi, t)$ --- положение частицы $\pi$ в момент времени $t$.
Сумма весов всех частиц изначально равна единице. Расширение понятия веса частиц на непрерывный
 случай аналогично расширению для вероятностного пространства.
``Собрать частицу'' --- проверить гипотезу. После проверки частица исчезает и больше не может
быть собрана.
Распределение частиц $f(dS, t, path)$ --- функция, ставящая в соответствие области и
 времени сумму весов частиц находящихся в этой области в заданное время,
 с учетом частиц собранных средством поиска к данному моменту. 
С течением времени частицы могут перемещаться в любом направлении с ограниченной скоростью, однако
не могут появляться из ниоткуда или исчезать иным способом, кроме сбора их средством поиска.
Сумма собранных частиц фактически равна априорной вероятности обнаружить объект, имеющий заданное
начальное распределение вероятности и закон его изменения.

\FloatBarrier
\subsection{Формулировка задачи построения маршрута поиска 
в общем случае}
Задано распределение частиц $f(dS, t, path)$, $t_{search}$ --- время поиска
и параметры средства поиска: $pos_0$ --- начальная позиция, $r$ --- радиус обнаружения и 
$v_{max}$ --- максимальная скорость передвижения.
 Необходимо найти маршрут $path(t)$ --- определяющий
позицию средства поиска в любой момент времени, максимизирующий сумму весов собранных
частиц $sum_{res}$.
Более формально $sum_{res}=1-\int\limits_Sf(dS, t_{search}, path)$, $path(0)=pos_0$,
 $|path'(t)|\le v_{max}$, собраны те и только те частицы
 $\pi$ для которых $\exists \tau\le t_{search} ||pos(\pi, \tau)-path(\tau)|| \le r$. 
\FloatBarrier
\section{Задача построения маршрута поиска
 стратегией ``Паралельное галсирование''}

\subsection{Стратегии поиска}

\begin{figure}[ht]
  \begin{center}
    \subfigure[``Гребенка'']{%
      \label{strat:comb}
      \putImgx{0.4\textwidth}{pic05-comb}
    }%
    \subfigure[``Расширяющийся квадрат'']{%
      \label{strat:box}
      \putImgx{0.4\textwidth}{pic05-expand_box}
    }\\ %--------------- End of first row -------------
    \subfigure[``Параллельное галсирование'']{%
      \label{start:tacks}
      \putImgx{0.4\textwidth}{pic05-parallel_tacks}
    }%
  \end{center}
 \caption{Стратегии поиска}
 \label{strat:subfigures}
\end{figure}

\FloatBarrier
\subsection{Формулировка задачи построения маршрута поиска стратегией
``Параллельное галсирвоние''}

\FloatBarrier
%%% Local Variables:
%%% mode: latex
%%% TeX-master: t
%%% End:

%-*-coding: utf-8-*-
\chapter{Решение массовой задачи о ближайшем отрезке с использованием квадродерева}
\section{Квадродерево}
Квадродерево -- поисковая структура данных, которая хранит в себе
подразбиение плоскости и позволяет быстро производить локализацию
точек-запросов. Узел (ячейка) квадродерева представляет собой прямоугольник, для
которого определена некоторая мера его насыщенности $p(C)$. Если узел
насыщен ($p(C) > T$, где $T$ -- предельное насыщение), то происходит его
разбиение на четыре одинаковых дочерних узла (делением пополам по
вертикали и по горизонтали). Таким образом, в каждый момент времени у узла
или нет детей, или их четыре. Разбиение происходит до тех пор, пока все узлы
не перестанут быть насыщенными, или не будет достигнута максимальная
глубина подразбиения. Ограничение глубины подразбиения играет важную
роль в виду того, что не всегда получается сделать узел ненасыщенным за
конечное (или разумное) количество разбиений, далее будет дано более 
точное обоснование необходимости ограничения.

Квадродеревья и их модификации очень часто применяют для решения
задач примерного поиска ближайшего соседа (Approximate Nearest Neighbor
Search) для точек \cite{FANN} (рис. \ref{ann}). В качестве меры насыщения в этой задаче часто
выбирают количество точек, среди которых производится поиск, попавших в
ячейку. Насыщенность обычно ограничивают одной точкой в одной ячейке.
Максимальная глубина древа для $n$ точек может составлять $n$, в результате чего
время локализации может составлять $O(n)$. Для борьбы с этим была разработана
структура данных Skip-Quadtree \cite{SQT}, которая позволяет производить локализацию за $O(\log n)$.

\drawfigurex{ann}{Квадродерево}{ann}{width=5cm}

Квадродерево обладает свойствами, которые делают его предпочтительным
для решения задачи dist2segments, по сравнению с другими деревьями.
Например, квадродерево не требует наличия логики, по которой будет происходить
разбиение ячейки (в отличии от kd-дерева). Разбиение всегда происходит на четыре
равные ячейки. Также важным свойством является быстрота локализации точки в нем.
За счет того, что квадродерево имеет регулярную структуру, можно по координатам точки
и глубине сразу получить путь в квадродереве. Более того, в виду замечательного
совпадения, можно за несколько операций над числами с плавающей точкой по координате точки получить путь
до листа. Остановимся на этом подробнее.

Рассмотрим устройство чисел с плавающей точкой. Они состоят из знака $s$, мантиссы $m$ 
и экспоненты $e$, причем $m \in [1, 2)$ и первая единица не хранится. Само число 
представляется в виде $sm2^e$. В компьютерном представлении экспонента хранится в виде
двоичной последовательности $b_1b_2\ldots b_n$, при этом $m = 1 + \sum\limits_{i=1}^nb_i2^{-i}$.

Теперь вернемся к квадродереву. Пусть $(x_0, y_0)$ -- координаты левого нижнего угла первого уровня квадродерева, 
а $w, h$ -- его ширина и высота. Перейдем в новую систему координат.\\
$
\left\{
\begin{array}{l}
x' = (x - x_0)/w  \\
y' = (y - y_0)/h  \\
\end{array}
\right.$

В этой системе координат внутренние точки квадродерева имеют координаты
$(x', y') \in [0, 1] \times [0, 1]$. Посмотрим, что происходит при переходе на уровень вниз.
В системе координат ячейки ее дети имеют координаты $[0, \frac{1}{2}] \times [0, \frac{1}{2}]$
-- левый нижний, $[\frac{1}{2}, 1] \times [0, \frac{1}{2}]$
-- правый нижний, $[0, \frac{1}{2}] \times [\frac{1}{2}, 1]$
-- левый верхний, $[\frac{1}{2}, 1] \times [\frac{1}{2}, 1]$
-- правый верхний. Опишем переход в систему координат ребенка.\\
$
\left\{
\begin{array}{l}
x_{i+1} = 2(x_i - x_{c_i})  \\
y_{i+1} = 2(y_i - y_{c_i})  \\
\end{array}
\right.$
\\

Заменим координаты левых нижних углов детей $(x_{c_i}, y_{c_i}) \in \{0, \frac{1}{2}\} \times \{0, \frac{1}{2}\}$
на $(b^x_{c_i}, b^y_{c_i}) = (2x_{c_i}, 2y_{c_i}) \in \{0, 1\} \times \{0, 1\}$.\\
$
\left\{
\begin{array}{l}
x_{i+1} = 2x_i - b^x_{c_i}  \\
y_{i+1} = 2y_i - b^y_{c_i}  \\
\end{array}
\right.$
\\
Теперь посмотрим на обратный переход.\\
$
\left\{
\begin{array}{l}
x_i = \frac{1}{2}(b^x_{c_i} + x_{i+1})  \\
y_i = \frac{1}{2}(b^y_{c_i} + y_{i+1})  \\
\end{array}
\right.$
\\
Найдем формулу перехода от уровня $d$ к самому верхнему уровню.\\
$
\left\{
\begin{array}{l}
x_1 = \sum\limits_{i=1}^db^x_{c_i}2^{-i} + x_d  \\
y_1 = \sum\limits_{i=1}^db^y_{c_i}2^{-i} + y_d  \\
\end{array}
\right.$
\\
Что по сути является двоичным представлением координат точки в системе координат верхней ячейки квадродерева.
То есть, совершив преобразование координат, мы получаем путь в квадродереве в виде битов
двоичного представления координат.

Эта техника позволяет производить быструю локализацию в квадродереве. 
Специальной обработки требует только экспонента. Также неоценимым достоинством этого метода
является его робастность. Если аккуратно произвести преобразование координат мы можем
получить путь на глубину $d = 53$ для чисел с плавающей точкой двойной точности.

\FloatBarrier
\section{Нижняя огибающая}
Неформально нижняя огибающая (lower envelope) множества объектов на плоскости –
множество точек этих объектов, видимых наблюдателем, расположенным в
точке $(0, -\infty)$. Формально же это граф, представляющий из себя поточечный
минимум кусочно-заданных функций \cite{LENV} (рис. \ref{lenv}).
Также наряду с нижней огибающей часто рассматривается минимизационная диаграмма
(minimization diagram), которая представляет собой проекцию нижней огибающей на
горизонтальную ось (рис. \ref{mdiag}). 
\drawfigure{lenv}{Нижняя огибающая}{lenv}
\drawfigure{mdiag}{Минимизационная диаграмма}{mdiag}

\FloatBarrier
\section{Алгоритм}
\subsection{Идея алгоритма}
Основной идеей всех алгоритмов поиска ближайших сайтов (sites),
основанных на подразбиении пространства, является растеризация (с явным
построением или без него) диаграммы Вороного в этом подразбиении. После
этого в ячейках подразбиения оказывается информация, обо всех ближайших
сайтах для всех точек этой ячейки. Поиск ближайшего сайта происходит
путем локализации в этом подразбиении, и последующим перебором всех
сайтов, ближайших к найденной ячейке.

Ввиду нетривиальности задачи поиска всех сайтов ближайших к ячейке,
во многих алгоритмах переходят к неточному решению задачи поиска
ближайшего отрезка \cite{NGRID}, производя поиск сайтов, ближайших к каким-то
точкам ячейки. Точки обычно выбираются таким образом, чтобы обеспечить
заданную точность, но в некоторых случаях даже не идет речи о точности \cite{AVOR}.
Для некоторых случаев погрешность допустима, но робастность (robustness)
является важной характеристикой алгоритмов вычислительной геометрии \cite{ROBUS}.
Предложенный алгоритм позволяет произвести точный поиск ближайших
отрезков для ячеек, при условии, что можно явно (хотя бы кусочно) задать
расстояние от границ ячеек до отрезков в виде полинома.

\FloatBarrier
\subsection{Работа алгоритма}

Для отрезков строится ограничивающий прямоугольник (bounding box), этот прямоугольник будет первым
уровнем квадродерева. В качестве меры насыщенности узла берется количество
ближайших отрезков к данной ячейке. Для первого узла ближайшими будут все
отрезки, так как они все лежат внутри. Далее происходит рекурсивное
подразбиение узлов. Ближайшие к дочернему узлу отрезки будут среди ближайших
к его родителю, так как дочерний узел геометрически лежит внутри
родительского. Необходимо произвести фильтрацию лишних отрезков.

{\prop\label{cl_segs}
Ближайшие отрезки для точек ячейки -- это отрезки ближайшие к ее границе, и отрезки пересекающие ячейку}
\begin{proof}
Обозначим: $S$ -- множество отрезков, ближайших к ячейке, $S_b$ -- ближайших к границе, $S_i$ --
пересекающих ячейку.
\begin{itemize}
\item $S_b \cup S_i \subset S$ \\
$S_b \subset S$ -- очевидно, так как граница ячейки -- это ее подмножество.\\
Для любой точки на пересечении отрезка и ячейки этот отрезок будет
ближайшим, значит $S_i \subset S$
\item $S_b \cup S_i \supset S$ \\
Предположим, что это не так.\\Пусть $s$ -- отрезок, не пересекающий
ячейку, и он не является ближайшим ни к одной точке на границе. Пусть
он ближайший для точки $P$ ячейки, а $Q$ -- точка $s$, ближайшая к $P$.
Построим отрезок $PQ$, так как точка $P$ вне ячейки, а точка $Q$ внутри, то $PQ$
пересечет границу ячейки, допустим в точке $F$. Рассмотрим отрезок $s'$,
ближайший к $F$. Пусть точка $E$ -- ближайшая точка на нем к $F$.
Так как $s'$ ближайший к $F$, то $|FE| < |FQ|$, по неравенству треугольника $|PF|
+ |FE| < |PE|$. Подставив первое неравенство во второе, мы получим, что
отрезок $s'$ ближе к $P$ чем $s$ (рис. \ref{contrex}).\\
Противоречие.
\end{itemize}
$\qedsymbol$
\drawfigure{contrex}{Противоречие}{contrex}
\end{proof}

Это простое утверждение показывает, что для фильтрации нам
необходимо взять из отрезков только те, которые являются ближайшими для
границы ячейки, и те, которые ее пересекают.
Для поиска отрезков, ближайших к границе ячейки, для каждой стороны
прямоугольника строится нижняя огибающая функций кратчайшего расстояния от
стороны до отрезков, которые фильтруются (рис. \ref{le_dist}).

Функция кратчайшего расстояния до отрезка состоит из трех частей: двух функции расстояния от
стороны до концов отрезка, и функции расстояния от стороны до прямой,
содержащей этот отрезок, заданной на ограниченном промежутке. В результате
из нижней огибающей можно выделить информацию об отрезках ближайших к
сторонам ячейки. Также эта фильтрация оставляет отрезки, пересекающие
границу. Значит, к полученным отрезкам остается только добавить отрезки
лежащие внутри ячейки. Для проверки этого условия достаточно проверить
принадлежность одной из точек отрезка ячейке.

\drawfigurex{le_dist}{Нижняя огибающая функций кратчайших расстояний}{le_dist}{width=6cm}

Подразбиение будет происходить до тех пор, пока все ячейки не
перестанут быть насыщенными, или пока не будет достигнута максимальная
глубина подразбиения. В данной задаче очень важно ограничить глубину
подразбиения, так как в вырожденных случаях (degenerate cases) некоторые
ячейки подразбить не получится. Вырожденным случаем для диаграммы
Вороного является наличие четырех и более сайтов равноудаленных от одной
точки. В таком случае в этой точке получается вершина диаграммы Вороного,
граничащая с ячейками соответствующих сайтов. В результате наличия
большого числа сайтов расположенных таким образом (пусть их $n$), ячейка
квадродерева, содержащая эту точку, будет ближайшей как минимум к $n$
сайтам. Разбив такую ячейку мы все равно получим одну ячейку содержащую
эту вершину диаграммы Вороного. Поэтому имеет смысл ограничивать
глубину разбиения.
Итак, в результате получается квадродерево, в листьях которого лежит
информация о ближайших к ним отрезках. Поиск ближайшего отрезка по такой
структуре данных осуществляется в два этапа. Сначала происходит
локализация точки-запроса в квадродереве. Затем перебираются все отрезки,
ближайшие к найденной ячейке, и среди них выбирается ближайший.

\FloatBarrier
\subsection{Оценка числа перебираемых отрезков}
Скорость обработки запроса очень сильно зависит от числа отрезков в ячейке.
Рассмотрим равномерное распределение точек-запросов на прямоугольнике,
задаваемом верхним уровнем квадродерева. Пусть $X$ -- случайный запрос, 
$n(X)$ -- число перебираемых отрезков, при запросе $X$, $C$ -- прямоугольник, 
покрываемый верхним уровнем квадродерева, $L$ -- множество листьев квадродерева.
\begin{equation}
E\{n(X)\} = \int\limits_Cn(X)dp(X) =  \sum\limits_{l \in L}n_lp_l
\end{equation}
Так как распределение равномерное, то $p_l = \frac{S_l}{S}$, где 
$S_l$ -- площадь листа $l$, $S$ -- площадь покрываемая квадродеревом.
В итоге получаем простую формулу.
\begin{equation}
E\{n(X)\} = \frac{1}{S}\sum\limits_{l \in L}n_lS_l
\end{equation}

Попытаемся оценить эту величину сверху. Пусть глубина дерева ограничена числом $d$, 
а $c$ -- насыщенность узла квадродерева, после которой он разбивается.
Заметим, что листья бывают двух видов: насыщенные и ненасыщенные. Насыщенные листья -- это
листья, которые находятся на уровне $d$ и все еще содержат больше, чем $c$ отрезков.
Обозначим множество ненасыщенных листьев $G$ (good), множество насыщенных листьев $B$ (bad).
\begin{equation}
E\{n(X)\} = \frac{1}{S}\sum\limits_{l \in L}n_lS_l = 
\label{expectation}
\frac{1}{S}\sum\limits_{l \in G}n_lS_l 
+
\frac{1}{S}\sum\limits_{l \in B}n_lS_l
\end{equation}
Оценим первую сумму.
\begin{equation}
\frac{1}{S}\sum\limits_{l \in G}n_lS_l \le \frac{c}{S}\sum\limits_{l \in G}S_l \le \frac{c}{S}S = c
\label{sum1}
\end{equation}
Оценим вторую сумму. Так как листья из $B$ находятся на уровне $d$, то $S_l = \frac{S}{4^d}$.
В каждом таком листе не более $n$ отрезков.
\begin{equation}
\frac{1}{S}\sum\limits_{l \in B}n_lS_l \le \frac{Sn}{4^dS}|B| = \frac{n|B|}{4^d}
\label{sum2}
\end{equation}
Осталось оценить мощность множества $B$. Ячейка попадает в множество $B$, если ее пересекает
большое число ячеек диаграммы Вороного исходного множества отрезков.
Это возможно в случае попадания туда вершины диаграммы Вороного большой степени (рис. \ref{big_deg})
или пересечения большим числом узких локусов (рис. \ref{thin_loc}). Вершин в диаграмме Вороного
$O(n)$, локусов тоже $O(n)$. 

\drawfigurex{big_deg}{Вершина диаграммы Вороного большой степени}{big_deg}{width=5cm}
\drawfigurex{thin_loc}{Узкие локусы}{thin_loc}{width=5cm}
\FloatBarrier

Рассмотрим подразбиение прямоугольника, покрываемого верхним уровнем 
диаграммы Вороного, на $4^d$ ячеек ($2^d$ по вертикали и горизонтали).
Границы ячеек диаграммы Вороного представляют из себя отрезки прямых и парабол, следовательно
узкие ячейки растеризуются в этой сетке как отрезки или параболы, а не как площадные объекты, 
так как они уже ячеек сетки. При растеризации в сетку, параболы и отрезки пересекают $O(n)$ ячеек (сетка $n \times n$).
Следовательно узкий локус пересекает $O(2^d)$ ячеек. Можно оценить $|B|$.
\begin{equation}
|B| = O(n2^d)
\label{bad_segs}
\end{equation}

Сводя все воедино, получаем верхнюю оценку математического ожидания числа перебираемых отрезков.
\begin{equation}
E\{n(X)\} = c + \frac{nO(n2^d)}{4^d} = c + \frac{O(n^2)}{2^d}
\end{equation}

Отсюда видно, что при $d = 2\log_2 n + C$, где $C$ -- константа, $E\{n(X)\} = O(1)$.

\FloatBarrier
\subsection{Анализ полученных результатов}
Если произвести грубую оценку времени построения квадродерева для
этой задачи, то получается, что оно ограничено только максимальной глубиной
подразбиения. Это так, но на практике построение происходит достаточно
быстро, в виду того, что сильное подразбиение испытывают в основном
области, содержащие вершины и узкие локусы диаграммы Вороного. 
Максимально возможное число ячеек будет $4^d$, что равно $O(n^4)$.

Время поиска ближайшего отрезка складывается из времени
локализации и времени поиска ближайшего отрезка среди ближайших к
ячейке. Тогда как первая величина ограничена сверху $d$, вторая ограничивается
только количеством отрезков (достаточно вспомнить вырожденный случай).
Ввиду того что $d$ обычно не очень велико и локализация в дереве, как
было показано, не требует на каждом шаге операций с плавающей точкой, 
а происходит простой спуск по заданному пути, основной вклад во время 
поиска ближайшего вносит перебор отрезков из ячейки.
Хоть локализация и занимает $O(\log n)$ времени, но, по уже оговоренным причинам,
алгоритм на разумном числе отрезков оказался очень эффективным, в виду того,
что математическое ожидание перебираемых отрезков $O(1)$. Практические данные
показывают, что эта величина лежит в промежутке $[0.5c, c]$, где $c$ -- предельная
насыщенность ячейки.

Максимальное количество занимаемой памяти можно грубо оценить, как
$O(2^d n + 4^d) = O(n^4)$, что на практике не наблюдается.

\FloatBarrier

%-*-coding: utf-8-*-
\chapter{Алгоритм построения маршрутов согласно стратегии
``Параллельное галсирование''}
\section{Алгоритм построения маршрута при
 фиксированном распределении}
\FloatBarrier

\section{Корректировка маршрута}
\FloatBarrier

%\chapter{Сравнение с существующими решениями}
\section{Обзор существующих решений}
Был разработан алгоритм строящий маршруты поиска, которые удовлетворяют фиксированным паттернам
стратегии ``Параллельное галсирование''. Задача построения маршрутов имеющих определенную
структуру достаточно специфична. К сожалению, методы, решающие поставленную задачу (или
ее частные случаи) не были найдены в открытых источниках.

 Однако существует комплекс инструментов, ранее разработанный в ``Кронштадт Технологии'',
который предоставляет возможность строить маршруты разными стратегиями поиска, несколькими
поисковыми средствами, оптимизировать параметры совместного поиска и некоторые другие возможности.
Рассмотренный в данной работе метод должен прийти на замену существующему в данном комплексе,
поэтому в первую очередь должно быть осуществлено сравнение с этим методом.

Далее будет произведено сравнение с существующим алгоритмом построения маршрутов стратегией
``Параллельное галсирование''.

\FloatBarrier
\section{Сравнение со старым алгоритмом построения маршрутов ``Кронштадт Технологии''}
Старый алгоритм имеет немного другой интерфейс. На вход принимается район в котором необходимо
осуществить галсирование. Единственная поддерживаемая модель изменения распределения ---
случайное блуждание с определенной интенсивностью $I$. Интенсивность $I$ определяет сколько
времени потребуется частицам, чтобы не менее $p\%$ отдалились на единицу длины. Исходя из
интенсивности рассчитывается расстояние между галсами --- одинаковое на всем протяжении маршрута.
Причем допускается пропуск частиц суммарного веса $\epsilon$ на каждом галсе.

Сравнение будет происходить следующим образом. Тест 1 продемонстрирует несостоятельность старого
метода в условии другой модели изменения распределения. Тест 2 покажет преимущества нового метода
при неравномерных распределениях. Все последующие тесты будут иметь равномерное распределение
и модель изменения --- случайное блуждание. Тест 3 покажет важность корректировки маршрута со
временем, так как район расширяется и ширина галса со временем должны быть увеличена.
Тест 4 проверит способность нового метода подстраивать расстояние между галсами в простейшем случае,
когда частицы не передвигаются. Кроме того тест 4 покажет способность нового метода учитывать зоны,
досматриваемые во время движения вверх.

В старом методе время поиска не фиксируется, а возвращается по результату работы алгоритма.
Таким образом для каждого теста новым алгоритм будет построено два маршрута. Первый маршрут
будет демонстрировать лучший результат, занимая столько же времени сколько маршрут
старого алгоритма. Если времени данного старым алгоритмом недостаточно для построения
хорошего маршрута в данном случае --- будет построен маршрут с большим временем поиска.
Второй маршрут будет демонстрировать лучшее время, при этом не теряя в результативности
по сравнению со старым алгоритмом.

Для каждого маршрута будет показана статистика --- прогресс и поисковая производительность
на протяжении всего времени поиска. Синий график поисковой производительности имеет
относительный масштаб и может быть использован только для сравнения поисковой производительности
в рамках одного поиска (если его абсолютная величина на одном из двух графиков меньше --- это
не значит, что поисковая производительность в этом запуске хуже чем во втором).
\subsection{Teст 1}

Воспользуемся моделью изменения распределения --- притяжение к определенным точкам. Пусть
частицы стремяться приблизиться к берегу, расположенному слева от района поиска. Новый алгоритм
скорректирует маршрут, в то время как старый продолжит обследовать область их начально
местоположения.

 Старый алгоритм запросил 3.6 часа и собрал 85.7\% суммарного веса.
За это же время новый алгоритм собрал 94.7\%, а для сбора 85.7\% ему потребовалось
менее 2.9 часа.

\subsection{Teст 2}
Модель изменения распределения --- случайное блуждание. В левом нижнем и правом верхнем
углу поля расположены нормальные распределения. Вес частиц значительно убывает при удалении
от центров распределений, следовательно поиск в тех местах не является целесообразным.
Старый алгоритм этого не учитывает.

Таким образом старый алгоритм по прежнему тратит 3.6 часа, собирая при этом 83.3\%.
За это же время более аккуратный новый алгоритм собирает 92.6\%. Для сбора старого
результата новому будет достаточно почти вдвое меньшего времени 1.9 часа.

\subsection{Teст 3}
Модель изменения распределения --- интенсивное случайное блуждание. Таким образом
район увеличивается со временем и построенный изначально маршрут будет покрывать все меньше
частиц с течением времени.

Старый алгоритм за 3.6 часа собирает 83.3\%, проходя по наиболее вероятным местам. Новый алгоритм
за то же время может собрать такой же вес. Однако если увеличить время поиска для нового алгоритма, он
начнет собирать частицы, отдаляющиеся от начального района.

\subsection{Teст 4}
В данном тесте частицы не передвигаются со временем. Старый алгоритм, ожидаемо, рассчитывает
верное расстояние между галсами и собирает все частицы за время 3.6 часа. За это же время
новый алгоритм при разрешении по строкам $T_H$, правильно выбрав расстояние, также собирает
все частицы. Однако с незначительными потерями (0.3\%) новый алгоритм может сэкономить
много времени на поворотах (до четверти часа), учитывая то, что на поворотах частицы также собираются.

 
\def\testPics#1#2{
\begin{figure}[ht]
  \begin{center}
    \subfigure[Старый алгоритм]{%
        \putImgx{0.4\textwidth}{t#1op}
        \label{t#1op}
    }
    \subfigure[Статистика]{%
        \putImgx{0.4\textwidth}{t#1os}
        \label{t#1os}
    } \\
    \subfigure[Новый --- сравнение по результату]{%
        \putImgx{0.4\textwidth}{t#1np}
        \label{t#1np}
    }
    \subfigure[Статистика]{%
        \putImgx{0.4\textwidth}{t#1ns}
        \label{t#1ns}
    } \\
    \subfigure[Новый --- сравнение по времени]{%
        \putImgx{0.4\textwidth}{t#1sp}
        \label{t#1sp}
    } 
    \subfigure[Статистика]{%
        \putImgx{0.4\textwidth}{t#1ss}
        \label{t#1ss}
    } 
\end{center}
  \caption{#2}
\end{figure}
}

\testPics{1}{Тест 1. Модель изменения --- приближение к прямой слева}
\testPics{2}{Тест 2. Начальное распределение --- композиция нормальных}
\testPics{3}{Тест 3. Случайное блуждание --- район увеличивается со временем}
\testPics{4}{Тест 4. Частицы не передвигаются с течением времени}

\FloatBarrier

\subsection{Результаты}
Далее приведена сводная таблица результатов тестов. Во втором столбце указано время,
которое занимает прохождение маршрута, построенного старым алгоритмом. В третьем столбце указан
результат, получаемый при прохождении старого алгоритма. В четвертом столбце указан
результат нового алгоритма при оптимизации по результату. Если не указано значения времени
в скобках --- значит данный результат получен за то же время что и результат старого алгоритма.
В последнем столбце указан результат нового алгоритма при минимизации времени. Если не указано
значения в скобках --- значит за данное время получен результат не хуже результата старого
алгоритма.

\begin{table}[ht]
  \centering
  \begin{tabular}{|l|l|l|l|l|}
\hline
    Номер & Время старого & Результат старого & Сравнение по & Сравнение по \\
    теста     & & & результату & времени \\
\hline
    1 & 3.6 & 87.5\% & 94.7\% & 2.9 \\ 
    2 & 3.6 & 83.3\% & 92.6\% & 1.9 \\
    3 & 3.6 & 83.3\% & 86.1\% (4.5) & 3.6 \\
    4 & 3.6 & 100.0\%& 100.0\%& 3.3 (99.7\%) \\
\hline

  \end{tabular}
  \captionsetup{justification=centering}
  \caption{Сводная таблица результатов работы на тестах}
  \label{results}
\end{table}

\FloatBarrier
%\startconclusionpage

В данной работе был разработан алгоритм построения маршрутов поиска стратегией
``Параллельное галсирование''. Алгоритм позволяет работать с неравномерными
распределениями и различными моделями изменения распределений со временем.

В качестве недостатков алгоритма, которые подлежат дальнейшему улучшению,
нужно отметить большое время работы при максимальном разрешении сетки
по строкам $T_H$. Хотя такое разрешение необходимо для наиболее точной
подстройки расстояния между галсами, на практике удается
обойтись меньшим разрешением без значительных потерь в результативности, при
этом время работы снижается до приемлемого уровня.

Для целей визуализации и в качестве вспомогательного средства для корректировки
маршрутов со временем был разработан симулятор прохождения маршрутов.
Симулятор основан на технологии CUDA, таким образом при высоком разрешением
поля симуляции удается поддерживать достаточную скорость работы для
визуализации прохождения маршрута в ускоренном режиме.

В качестве дальнейших улучшений симулятора можно рассмотреть повышение
разрешения посредством использования более продвинутых техник осуществления
свертки изображения (Image Convolution) на GPU.

Разработанный алгоритм должен заменить существующий алгоритм
``Кронштадт технологии'', сравнение с которым было осуществлено
в главе 4. Было показано, что разработанный алгоритм не уступает старому и
значительно превосходит его в случае неравномерных распределений и
моделей их изменения отличных от случайных блужданий. 

\FloatBarrier


%\startappendices
%\label{appendix}
%\input{appendix}

\bibliographystyle{sty/utf8gost705u}
\bibliography{thesis}

\end{document}

%%% Local Variables:
%%% mode: latex
%%% TeX-master: t
%%% End:
